    %!TEX TS-program = xelatex
    %!TEX encoding = UTF-8 Unicode
    \documentclass[12pt, letterpaper]{article}
    \usepackage{fontspec}
    \usepackage{placeins}
    \usepackage{hyperref}
    \usepackage{multibbl}
    \usepackage{graphicx}
    \usepackage{txfonts}
    \usepackage{geometry}
    \usepackage{hieroglf}
    \usepackage[all]{nowidow}

    \geometry{letterpaper, textwidth=5.5in, textheight=8.5in, marginparsep=7pt, marginparwidth=.6in}
    %\setlength\parindent{0in}
    \defaultfontfeatures{Mapping=tex-text}
    \setromanfont [Ligatures={Common}, SmallCapsFont={Adobe Jenson Pro}, BoldFont={Adobe Jenson Pro Bold}, ItalicFont={Adobe Jenson Pro Italic}]{Adobe Jenson Pro}
    \setmonofont[Scale=0.8]{Lucida Sans Typewriter Std}
    \setsansfont [Ligatures={Common}, SmallCapsFont={ITC Officina Sans Std}, BoldFont={ITC Officina Sans Std Bold}, ItalicFont={ITC Officina Sans Std Book Italic}]{ITC Officina Sans Std}
    % ---- CUSTOM AMPERSAND
    \newcommand{\amper}{{\fontspec[Scale=.95]{StoneSansStd-MediumItalic}\selectfont\itshape\&}}
    % ---- MARGIN TASK (year, task, etc.)
    \newcommand{\task}[1]{\marginpar{\small #1}}
    \usepackage{sectsty}
    \usepackage[normalem]{ulem}
    \sectionfont{\sffamily\mdseries\upshape\Large}
    \subsectionfont{\sffamily\mdseries\scshape\normalsize}
    \subsubsectionfont{\sffamily\mdseries\upshape\normalsize}
    \begin{document}
    \thispagestyle{empty}
    \reversemarginpar
    \noindent
    \includegraphics[scale=0.33]{/home/rosslaird/Documents/professional/ibis}\\[1em]
    {\LARGE Ross A. Laird, PhD}\\[1em]
    Phone: \texttt{604-916-1675}\\
    Email: \texttt{\href{mailto:ross@rosslaird.com}{ross@rosslaird.com}}\\
    \textsc{url}: \texttt{\href{http://www.rosslaird.com}{www.rosslaird.com}}\\
    \textsc{PGP key}: \href{http://keyserver.ubuntu.com:11371/pks/lookup?op=get&search=0x623D9CC650BD6C0B}{\texttt{50BD6C0B}}
    \\[4em]
\noindent
\today\\
To the Dean of Arts Teaching Award Selection Committee
\\[2em]
Dear colleagues;
\\[1em]
I am honored to be considered for this award. I know that there are many exemplary teachers at Kwantlen. In the notification email that I received about the award, the committee invited the nominees to submit further materials that highlight “aspects of your teaching and/or pedagogical approach you would like the committee to consider.” I am not sure, exactly, what might be of most use to the committee, so perhaps the best thing is just to describe what I do, and what I believe, as an educator.

The courses I teach at Kwantlen (in Interdisciplinary Expressive Arts and
Creative Writing) are focused on the integration of personal, professional, and
academic development. My approach emphasizes learner engagement, purposeful
creativity, and experiential self-development. I believe in the power of education as a transformative force.

The courses that I teach at Kwantlen  are specifically designed to help learners ask foundational questions and pursue meaningful answers. Through learner-designed projects, individualized curriculum, and many other creative approaches, these courses foster collaborative learning environments in which learners discover their own paths and purposes within and beyond the classroom. I endeavor to promote university experiences that are reflective, individualized, and joyful. And, perhaps most importantly, I recognize that beneath academic cultures and traditions lies the authentic search for knowledge, wisdom, and personal connection. I support learners in that search and encourage them to follow it — wherever it may lead.

My approach to teaching is founded upon three core values: self-awareness, empathy, and character. Self-awareness is what we know — about ourselves, our interests, our capacities. Empathy is what we feel — toward others, toward the state of our world, toward nature. Character is what we do with our self-awareness and our empathy. Awareness, emotion, behavior. Or, to put it another way: being, feeling, doing. These represent the core of my approach: the whole enterprise of teaching and learning rendered down to the alchemy between three foundational parts of ourselves. In my courses, all disciplines are one discipline: the search for self-awareness, empathy, and character.

In my classroom I try to help learners build an integrative and engaging learning environment. We explore many forms of creativity, we learn to facilitate and lead our interpersonal and creative process, and we examine our personal strengths and vulnerabilities. Our learning environment is collaborative, with the instructor serving as mentor and guide. We build upon the individual strengths of each participant to discover the quality and integrity of our learning community. Our classroom is built by learners and for learners. With self-awareness, empathy, and character development as the foundations of our curriculum and learning environment — with both learners and the instructor grounded in these values — we construct communities of real inquiry, creative engagement, and personal development.

I am proud of the leadership and achievement demonstrated by the many students with whom I have the privilege to work. They have won awards, have earned scholarships, and have presented at conferences. They consistently demonstrate leadership in academic life and beyond. The engaging environment I try to create in my classroom — with focus on the integration of personal and professional development — provides exemplary opportunities for learners to develop the skills necessary to achieve outstanding results in whatever they pursue. (In 2012, four of the finalists for — including the winner of — the Dean’s Medal in the Faculty of Arts were students who had taken at least one course in which I was the instructor.)

Overall, my philosophy of teaching is a big idea about how we might transform learning environments to be more engaging, purposeful, and fun. I try to respond to the yearning that many learners feel for university experiences that engage, provoke, and encourage, and for classroom environments that bring learners into authentic relationships with themselves and one another.

Ultimately, that’s what my teaching is all about: authentic relationships. For all of
us, at every age, the quality of our relationships defines the quality of our
lives — whatever we do academically or personally. I view my role as one of
service to that aim: to help learners find ways to increase the depth and
purpose of their relationships with themselves and others. Deepening relationships has been of primary interest to me beyond the classroom as well. I have worked with many colleagues in the Kwantlen community (faculty, learners, administrators) on interdisciplinary collaborations such as the Amazon Field School (a collaboration with the Faculty of Design), the Aboriginal mentorship program (a collaboration with Student Life), the re-imagined Boost Camps (with the Learning Centre), and the new Ecology and Creativity course (co-developed with the Faculty of Science). Collaboration, creativity, and collegiality are all foundational to my purpose as an educator.

I came to Kwantlen a few years ago, after more than 20 years of working as a
counsellor and teacher with populations at risk (addictions, homelessness,
disability, youth mental health), and the spirit of service that suffuses the
work of counselling has not left me. My view is that we must go where we are
called, to be of service to others in need, and I find at Kwantlen a great
number of students calling out for the kinds of services I provide.

Ultimately, the work that I do at Kwantlen is a call to service. I am honored to be called. For me, there is no more worthwhile professional path. Over the
years of my career, the shape of that service has taken various forms:
counselling with high-risk populations, teaching in the social services, and
consulting to a diverse range of educational and professional groups. When I
came to Kwantlen, much of that work was distilled into my teaching and into the various commitments I have made to Kwantlen’s organizational development. I have been involved in the creation of the mission and mandate, in the Foundations of Excellence (for which I was co-chair of the Learning Committee), in scenario planning, and in strategic planning. I am committed to being of service to Kwantlen and to seeing what we can achieve together.
\\[1em]
\noindent
Sincerely,\\[2em]
    \includegraphics[scale=0.20]{/home/rosslaird/Documents/professional/signature}

\noindent
Ross A. Laird, PhD\\

\end{document}