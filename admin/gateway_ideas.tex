\TeXXeTstate=1
\documentclass[12pt,DIV10,oneside,headsepline,letterpaper]{scrreprt}
\usepackage{fontspec,xunicode}
\defaultfontfeatures{Numbers=OldStyle,Scale=MatchLowercase,Mapping=tex-text}
\setmainfont{Sabon LT Std}
\setromanfont[Mapping=tex-text]{Adobe Jenson Pro}
\setsansfont[Scale=MatchUppercase,Letters=SmallCaps]{Myriad Pro}
\usepackage{graphicx}
\usepackage{xltxtra}
\setcounter{secnumdepth}{-1}
\begin{document}
\section{Collaborative Opportunities: Gateway and Kwantlen}

Based on shared values of creativity, community engagement, teaching and learning, and a shared vision of programming for arts and culture in the South Fraser region, Gateway Theatre and Kwantlen Polytechnic University are well-positioned to work together for mutual benefit. The services provided by each organization currently complement the other (for example, courses in arts and culture at Kwantlen complement arts and culture performances at Gateway). 

The arts and culture sectors, while always vibrant and dynamic, are also consistently challenged by economic and social factors that require a particular blend of adaptive and collaborative strategies. Accordingly, Gateway and Kwantlen would both be well-served by developing, conjointly, programs and initiatives that bring more participants and viewers to Gateway and a greater level of student engagement (and enrolment) to Kwantlen.

Collaborations might take several forms. Based on preliminary discussions, several options and ideas are provided below:

\subsubsection{Adult Classes in Creativity and Theatre}

Classes might be offered to university students working toward degrees in arts and culture, members of the general public wishing to explore their own creativity (and earning credit for it), and anyone searching for an interesting continuing education experience (offered, perhaps, through a collaborative effort between Gateway and Kwantlen's new Continuing Studies division).

Classes would be held at Gateway (in the current space, and in the new academy building when complete).

Adult classes in creativity and theatre represent a significant, essentially untapped market for Kwantlen (and a significant source of expanded revenue for Gateway). There is no university (or college) program south of the Fraser that acts as a community partner in theatre arts and cultural creativity. This seems like a strange oversight but is also a significant opportunity.

\subsubsection{Student Practica in Interdisciplinary Arts}

As Kwantlen develops further programming in arts and culture (particularly the Interdisciplinary Arts initiative), students will be increasingly required to complete practica in career-focused community experiences in their chosen fields of study. Students working toward careers in the arts would be ideally suited to volunteer placements at Gateway, where they would learn about -- and contribute to -- ongoing programming and development. A good example of such contribution might be a student who works with the development team for Gateway's expansion project. The student would help develop plans and proposals, would be involved with submissions and fund-raising, would assist with marketing, and might even contribute to new programming.

\subsubsection{A School for the Creative and Performing Arts?}

Kwantlen's Department of Creative Writing offers courses in writing for various genres (fiction, nonfiction, poetry, screenwriting, etc.). Other departments offer a wide range of courses in the arts and culture (the Department of Fine Arts, for example, and the Department of Music). Many of the courses offered by these departments might be offered in an intensive format at Gateway, as part of a creative collaboration in which Gateway becomes, essentially, a centre for creative and performing arts at Kwantlen. Kwantlen does not have such a centre, and this is a significant impediment to the development of creative programming and education at Kwantlen. For some time, Kwantlen has been considering the formation of a School for the Creative and Performing Arts, but this is a long-term initiative which will take several years (at least) to develop. In the interim, Gateway and Kwantlen might collaborate on a kind of pilot project, in which a nascent School for the Creative and Performing Arts develops out of work shared by, and developed in concert with, Gateway.

\subsubsection{Courses in Theatre and Creativity}

Two members of Kwantlen's Department of Creative Writing (Ross Laird and Aaron Bushkinsky) have extensive backgrounds, and strong interests in, theatre (Aaron) and creativity (Ross). The department would like to develop, in consultation with Gateway, a series of courses and workshops in writing for theatre, developing personal creativity, and learning the arts of interdisciplinary performance.

\bigskip
\textsection
\bigskip

These are just a few of the possibilities that exist. Many more might come about through thoughtful, creative planning.

\end{document}
