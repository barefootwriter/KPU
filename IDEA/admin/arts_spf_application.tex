\documentclass[]{article}
 \TeXXeTstate=1
   \usepackage{fontspec,xunicode}
    \defaultfontfeatures{Numbers=OldStyle,Scale=MatchLowercase,Mapping=tex-text}
    \setmainfont{Adobe Jenson Pro}
    \setromanfont[Mapping=tex-text]{Adobe Jenson Pro}
    \setsansfont[Scale=MatchUppercase]{Brioso Pro}
   \usepackage{graphicx}
   \usepackage{xltxtra}
\usepackage{ifxetex,ifluatex}
\usepackage{fixltx2e} % provides \textsubscript
% use microtype if available
\IfFileExists{microtype.sty}{\usepackage{microtype}}{}
% use upquote if available, for straight quotes in verbatim environments
\IfFileExists{upquote.sty}{\usepackage{upquote}}{}
\ifnum 0\ifxetex 1\fi\ifluatex 1\fi=0 % if pdftex
  \usepackage[utf8]{inputenc}
\else % if luatex or xelatex
  \usepackage{fontspec}
  \ifxetex
    \usepackage{xltxtra,xunicode}
  \fi
  \defaultfontfeatures{Mapping=tex-text,Scale=MatchLowercase}
  \newcommand{\euro}{€}
\fi
\ifxetex
  \usepackage[setpagesize=false, % page size defined by xetex
              unicode=false, % unicode breaks when used with xetex
              xetex]{hyperref}
\else
  \usepackage[unicode=true]{hyperref}
\fi
\hypersetup{breaklinks=true,
            bookmarks=true,
            pdfauthor={},
            pdftitle={Arts SPF Application},
            colorlinks=true,
            urlcolor=blue,
            linkcolor=magenta,
            pdfborder={0 0 0}}
\urlstyle{same}  % don't use monospace font for urls
\setlength{\parindent}{0pt}
\setlength{\parskip}{6pt plus 2pt minus 1pt}
\setlength{\emergencystretch}{3em}  % prevent overfull lines
\setcounter{secnumdepth}{0}

\title{Arts SPF Application}
\author{}
\date{}

\begin{document}
\maketitle

Name of Applicant: Ross Laird

Department or Program Affiliation: Interdisciplinary Expressive Arts

Email Address:
\href{mailto:ross.laird@kwantlen.ca}{ross.laird@kwantlen.ca}

Funding Category: D (Program Innovation \& Expansion)

Title of Project/Proposal: \\
Post-baccalaureate/professional education
credential development

Total Budget Requested from Arts SPF: \$2500

Dates of the initiative: September to December, 2013

\section{Project Description}

\subsection{Purpose}

Interdisciplinary Expressive Arts (IDEA) is in the process of developing
a post-baccalaureate, professional education certificate in Mentorship.
This certificate will be designed for a wide range of professionals
(teachers, counsellors, social service providers, life coaches, medical
practitioners, etc.) whose work depends upon effective mentorship. The
skills and practices of mentorship -- which are different from those of
counselling, coaching, or teaching -- are already foundational to all of
the Interdisciplinary Expressive Arts courses at Kwantlen, and we are
now in the process of adapting these courses for the specific cohort of
post-baccalaureate and professional learners. We are now beginning the
work of external consultations, assessment of learner demand, adaptation
of curriculum, and exploration of external accreditation opportunities.
We now seek funding support to facilitate these activities.

\newpage
\subsection{Main Elements}

We seek support in the following areas:

\begin{enumerate}
\def\labelenumi{\arabic{enumi}.}
\item
  Interviewing potential learners in various fields. This work requires
  travel to where people work, to see what they do, and to discuss with
  them ways in which the mentorship certificate will help them improve
  the quality of their work and their relationships.
\item
  Adapting curriculum. Language adjustments are required for some IDEA
  courses so as to position those courses specifically for
  post-baccalaureate learners. We also need to explore ways of creating
  a more efficient and laddered framework so as to conjoin the courses
  more tightly and to deliver a more integrated program to professional
  learners. And, as post-baccalaureate and professional education
  programs do not yet exist within the Faculty of Arts, we need to
  consult with our colleagues to develop pathways for administration,
  development, funding, and so on. There is no path yet for this type of
  initiative; we'll need to build it as part of this process.
\item
  Consulting with post-secondary institutional peers. We would like to consult with similar programs (at the University of the Fraser Valley, the University of Calgary, and other international institutions) and to seek their assistance in our own credential development (exploring themes such as credits, credentialing, integration with existing programs, and positioning with respect to professional studies and post-baccalaureate studies).
\item
  Reaching out to peers and possible partner institutions within the social services and arts fields. This work has already begun (with very positive
  responses) but requires further focus and commitment.
\end{enumerate}

The outcome of this developmental process will be the creation and submission of a formal proposal for the Certificate in Mentorship.


\subsection{Benefits to Arts}

In the new strategic plan, the Faculty of Arts ``commits to the development of
innovative and responsive new credentials in three key
revenue-generating areas: post-baccalaureate studies, continuing and
professional studies, and programs for international students.'' This
new credential applies to each of these areas and is an innovative and
responsive initiative. Additionally the IDEA Certificate in Mentorship
will showcase the Faculty of Arts as a place of innovation and creativity.

In the new strategic plan, the Faculty of Arts also indicates its intent to ``seek opportunities to expand peer and institutional mentorship for and among
students, alumni, faculty, staff, and administrators, e.g. lower and upper level student peer mentorship, chair mentorship, new faculty mentorship, service mentorship for faculty, teaching mentorship.'' The IDEA Certificate in Mentorship will facilitate all of these proposed directions.

This initiative is endorsed and supported by Jane Fee, Associate Vice
President Academic and Deputy Provost; by Diane Purvey, Dean of Arts;
and by the IDEA Steering Committee.

\subsection{Budget}

Travel for interviews (with potential learners) and
consultations (with potential partners): \$1000

Curriculum adaptation: \$1000

Proposal consultation and development: \$500

(Attendance at the National Mentoring Symposium is also planned, but those costs will be applied for under Professional Development guidelines.)

\end{document}
